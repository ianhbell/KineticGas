\section{Thermal diffusion}

Reiterating that the flux of species $i$ in the centre of mass frame of reference is given as
\begin{equation}
    \mflux_i = n_i (\bvu_i - \vu^m) = \int f_i \vU_i \d \vu_i.
    \label{eq:molar_flux_mass_for2}
\end{equation}
Investigate now the thermal response functions $\vA_i$, which are expanded in the Sonine polynomials as

\begin{equation}
    \vA_i = - \frac{m_i}{2 k_B T} \sum_{p = 0}^{\infty} a_i^{(p)} S_{3/2}^{(p)}(\sU^2)
\end{equation}

This section will first describe how the expansion coefficients $a_i^{(p)}$ are determined, before relating the coefficients of Equation \eqref{eq:molar_flux_with_temp} to the thermal diffusion coefficient ($D_T$). The relationship to the thermal diffusion ratio ($k_T$), thermal diffusion factor ($\alpha_T$) and Soret coefficient ($S_T$) are covered at the end of the section.

\subsection{Determining the expansion coefficients}

In order to satisfy the constraints posed by the summational invariants, the expansion coefficients $a_i^{(p)}$ must satisfy
\begin{equation}
    \begin{split}
        \sum_i \omega_i a_i^{(0)} &= 0\\
        \sum_j \sum_q \Lambda_{1j}^{(pq)}a_j^{(q)} &= \frac{4}{5k_B} x_1 K_1 \delta_{p1}, \hspace{1cm} p = \{1, 2, ..., N\} \\
        \sum_j \sum_q \Lambda_{ij}^{(pq)}a_j^{(q)} &= \frac{4}{5k_B} x_i K_i \delta_{p1}, \hspace{1cm} 
        \begin{cases}
            i = \{2, 3, ..., s\} & \\
            p = \{0, 1, ..., N\}
        \end{cases}
    \end{split}
\end{equation}

where $\omega_i$ denotes the weight fraction of species $i$, and $\Lambda_{ij}^{(pq)}$ are given by Equation \eqref{eq:lambda_ijpq}. This set of Equations may be written in matrix form in a manner analogous to that in Section \ref{sec:diffusion_exp_coeff},

\begin{equation}
    \Mat{\Lambda} \Vec{a} = \Vec{\lambda}
\end{equation}
where $\Mat{\Lambda}$ is a $Ns \times Ns$ matrix consisting of the blocks
\begin{equation}
    \Mat{\Lambda} = \begin{bmatrix} \Vec{\omega} \\ \Mat{\Lambda}_1^{(p > 0)} \\ \Mat{\Lambda}_{i > 1}\end{bmatrix}
\end{equation}
where 
\begin{equation}
    \Vec{\omega} = \begin{pmatrix} \omega_1 & \omega_2 & \hdots & \omega_s & 0 & \hdots & \times s(N - 1) & \hdots & 0 \end{pmatrix}
\end{equation}
\begin{equation}
    \Mat{\Lambda}_1^{(p > 0)} = 
    \begin{bmatrix}
        \Lambda_{1}^{(0)} & \Lambda_{1}^{(1)} & \hdots & \Lambda_{1}^{(N)}
    \end{bmatrix}
    ,\hspace{1cm}
    \Lambda_{1}^{(q)} = 
    \begin{bmatrix}
        \Lambda_{11}^{(1q)} & \Lambda_{12}^{(1q)} & \hdots & \Lambda_{1s}^{(1q)} \\
        \Lambda_{11}^{(2q)} & \Lambda_{12}^{(2q)} & \hdots & \Lambda_{1s}^{(2q)} \\
        \vdots & \vdots & \ddots & \vdots \\
        \Lambda_{11}^{(Nq)} & \Lambda_{12}^{(Nq)} & \hdots & \Lambda_{1s}^{(Nq)}
    \end{bmatrix}
\end{equation}
\begin{equation}
    \Mat{\Lambda}_{i>1} = 
    \begin{bmatrix}
        \Mat{\Lambda}_{i>1}^{(00)} & \Mat{\Lambda}_{i>1}^{(01)} & \hdots & \Mat{\Lambda}_{i>1}^{(0N)} \\
        \Mat{\Lambda}_{i>1}^{(10)} & \ddots             &        & \vdots \\
        \vdots             &                    & \ddots & \vdots \\
        \Mat{\Lambda}_{i>1}^{(N0)} & \hdots & \hdots  & \Mat{\Lambda}_{i>1}^{(NN)}
    \end{bmatrix}, \hspace{.5cm}
    \Mat{\Lambda}_{i>0}^{(pq)} = 
    \begin{bmatrix}
        \Lambda_{21}^{pq} & \Lambda_{22}^{pq} & \hdots & \hdots & \Lambda_{2s}^{pq} \\
        \Lambda_{31}^{pq} & \ddots & & & \vdots \\
        \vdots & & \ddots & & \vdots \\
        \Lambda_{s1}^{pq} & \hdots & \hdots & \Lambda_{s,(s-1)}^{pq} & \Lambda_{ss}^{pq}
    \end{bmatrix}
\end{equation}

$\Vec{a}$ and $\Vec{\lambda}$ are given as

\begin{equation}
    \Vec{a} = \begin{pmatrix}a_1^{(0)} \\ a_2^{(0)} \\ \vdots \\ a_s^{(0)} \\ a_1^{(1)} \\ \vdots \\ a_s^{(N)}\end{pmatrix}
    , \hspace{1cm}
    \Vec{\lambda} = \frac{4}{5 k_B} \begin{pmatrix} 0 \\ x_1 K_1 \\ 0 \\ \vdots \\ \times (N - 2) \\ \vdots \\ 0 \\ \Vec{K}_2 \\ \vdots \\ \Vec{K}_s \end{pmatrix}
    , \hspace{1cm}
    \Vec{K}_i =  \begin{pmatrix} 0 \\ \vdots \\ \times (i - 1) \\ \vdots \\ 0 \\ x_i K_i \\ 0 \\ \vdots \\ \times (N - i) \\ \vdots \\ 0 \end{pmatrix}
\end{equation}

\subsection{Force-Flux relations}

In the absence of a pressure gradient and external forces, the flux of species $i$ in the center of mass frame of reference may be expressed as
\begin{equation}
    \mflux_i = - \sum_{j \neq i} D_{ij}^{Fick} \frac{m_j}{m_i} \nabla n_j - \frac{\rho D_i^T}{m_i} \nabla \ln T.
    \label{eq:DTi_def}
\end{equation} 
Where $D_i^T$ are the thermal diffusion coefficients. Evaluating the integral of Equation \eqref{eq:molar_flux_mass_for2} to acquire an expression for the flux in terms of the expansion coefficients yields
\begin{equation}
    \mflux_i = \frac{x_i}{2} \lsp a_i^{(0)} \nabla \ln T - \sum_j d_{i, j}^{(0)} \Vec{d}_j \rsp
    \label{eq:molar_flux_with_temp}
\end{equation}

Inserting the definition of $\Vec{d}_j$ from Equation \eqref{eq:diffusion_driving_force} with $\nabla p = \Vec{F}_i = 0$ $\forall$ $i$, Equation \eqref{eq:molar_flux_with_temp} becomes
\begin{equation}
    \begin{split}
        \mflux_i &= \frac{x_i}{2} \lsp a_i^{(0)} \nabla \ln T - \sum_j d_{i, j}^{(0)} \sum_k x_j \lrp \delta_{j,k} + \frac{4 \pi}{3} n_k M_{jk} \sigma_{jk}^3 \chi_{jk}\rrp \nabla \ln T - \frac{1}{n} E_{jk} \nabla n_k \rsp \\
        &=  \frac{x_i}{2} \lsp a_i^{(0)} - \sum_j d_{i, j}^{(0)} \sum_k x_j \lrp \delta_{j,k} + \frac{4 \pi}{3} n_k M_{jk} \sigma_{jk}^3 \chi_{jk}\rrp \rsp \nabla \ln T - \frac{x_i}{2n} \sum_j d_{i, j}^{(0)} \sum_k E_{jk} \nabla n_k
    \end{split}
    \label{eq:mflux_com_expanded}
\end{equation}

In order to compare Equations \eqref{eq:DTi_def} and \eqref{eq:mflux_com_expanded} the gradient $\nabla n_i$ must be eliminated from the rightmost summation in Equation \eqref{eq:mflux_com_expanded}. Using the condition $\sum_k \vd_k = 0$, and denoting $b_{ij} \equiv \frac{4 \pi}{3} n_j M_{ij} \sigma_{ij}^3 \chi_{ij}$
\begin{equation}
    \begin{split}
        \vd_i &= - \sum_{k \neq i} \vd_k\\
        \sum_j x_i \lrp \delta_{i,j} + b_{ij}\rrp \nabla \ln T + \frac{1}{n} E_{ij} \nabla n_j &= - \sum_{k \neq i} \sum_j x_k \lrp \delta_{k,j} + b_{ij}\rrp \nabla \ln T + \frac{1}{n} E_{kj} \nabla n_j \\
        \frac{E_{ii}}{n} \nabla n_i + \frac{1}{n} \sum_{j \neq i} E_{ij} \nabla n_j \hspace{1cm}&\\
        + \sum_j x_i (\delta_{i, j} + b_{ij}) \nabla \ln T &= - \sum_{k \neq i} \left\{ \frac{E_{ki}}{n} \nabla \ln n_i + \frac{1}{n}\sum_{j \neq i} E_{kj} \nabla n_j \right.\\
        &\hspace{1.5cm} + \left. \sum_j x_k (\delta_{k,j} + b_{kj}) \nabla \ln T\right\} \\
        \frac{1}{n} \sum_k E_{ki} \nabla n_i &= - \sum_k \lcp \frac{1}{n} \sum_{j \neq i} E_{kj} \nabla n_j + x_k\sum_j (\delta_{k,j} + b_{kj}) \nabla \ln T \rcp \\
        \nabla n_i &= - \sum_{j \neq i} \frac{E_j'}{E_i'} \nabla n_j - \sum_k \frac{n_k}{E_i'} \sum_j (\delta_{k,j} + b_{kj}) \nabla \ln T
    \end{split}
\end{equation}

where the final equality is acquired by inverting the summation over $k$ and $j$, and using $E_i' \equiv \sum_j E_{ji}$. Inserting this expression back into Equation \eqref{eq:mflux_com_expanded} and collecting the terms related to each of the gradients yields
\begin{equation}
    \begin{split}
        \mflux_i &=  \frac{x_i}{2} \lsp a_i^{(0)} - \sum_j d_{i, j}^{(0)} \sum_k x_j \lrp \delta_{j,k} + b_{jk}\rrp \rsp \nabla \ln T \\
        & \hspace{1cm} - \frac{x_i}{2n} \sum_j d_{i, j}^{(0)} \lsp - E_{ji} \lrp \sum_{k \neq i} \frac{E_k'}{E_i'} \nabla n_k + \sum_k \frac{n_k}{E_i'} \sum_\ell (\delta_{k,\ell} + b_{k\ell}) \nabla \ln T \rrp + \sum_{k \neq i} E_{jk} \nabla n_k \rsp \\
        &= \frac{x_i}{2} \lsp a_i^{(0)} + \sum_j d_{i,j}^{(0)} \sum_k x_k \lrp \frac{E_{ji}}{E_i'} - \delta_{k,j}\rrp\sum_\ell \lrp \delta_{k,\ell} + b_{k\ell}\rrp \rsp \nabla \ln T \\
        & \hspace{1cm} - \frac{x_i}{2n} \sum_{k \neq i} \nabla n_k \sum_j d_{i,j}^{(0)} \lrp E_{jk} - E_{ji} \frac{E_k'}{E_i'}\rrp
    \end{split}
    \label{eq:mflux_com_expanded_no_ni}
\end{equation}

Comparing Equation \eqref{eq:molar_flux_with_temp} and \eqref{eq:mflux_com_expanded_no_ni} we can identify

\begin{equation}
    \begin{split}
        D_i^T &= - \frac{m_i x_i}{2 \rho} \lsp a_i^{(0)} + \sum_j d_{i,j}^{(0)} \sum_k x_k \lrp \frac{E_{ji}}{E_i'} - \delta_{k,j}\rrp\sum_\ell \lrp \delta_{k,\ell} + b_{k\ell}\rrp \rsp\\
        &= \frac{m_i x_i}{2 \rho} \lsp - a_i^{(0)} + \sum_j d_{i,j}^{(0)} \sum_k x_k \lrp \delta_{k,j} - \frac{E_{ji}}{E_i'} \rrp\sum_\ell \lrp \delta_{k,\ell} + b_{k\ell}\rrp \rsp
    \end{split}
\end{equation}

\subsection{Various measures of thermal diffusion}

Measuring the thermal diffusion coefficient $D_i^{T}$ is not necessarily trivial. Therefore, other measures of thermal diffusion and the relationship between them are presented here. We define the Soret coefficients ($S_{T,i}$) and thermal diffusion factors ($\alpha_{T,i}$) as

\begin{equation}
    S_{T,i} \equiv \frac{\nabla x_i}{x_i(1 - x_i) \nabla T}, \hspace{.5cm} \Vec{J}_j = 0 \ \forall \ j, \hspace{1cm} \alpha_{T,i} \equiv T S_{T,i}, \hspace{1cm}
\end{equation}

Where the lacking superscript on $\Vec{J}$ is intentional, as if all fluxes vanish in one frame of reference the same is true in all other frames of reference, by the transformation in Equation \eqref{eq:flux_transform_for}.

The thermal diffusion ratio, we define by the condition
\begin{equation}
    \mflux_i = 0\hspace{.5cm} \forall \hspace{.5cm} i \hspace{.5cm} \implies \hspace{.5cm} \nabla n_i = - n_i k_{T,i} \nabla \ln T, \hspace{.5cm} \forall \hspace{.5cm} i
\end{equation}

Inserting this definition into Equation \eqref{eq:DTi_def} yields a set of equations defining $k_{T,i}$
\begin{equation}
    \sum_{j \neq i} D_{ij} k_{T,j} \equiv \rho D_i^T, \hspace{1cm} \forall \hspace{.5cm} i.
\end{equation}

Rewriting the definition of $k_{T,i}$ in terms of the mole fraction gradients, one can relate $k_{T,i}$ to $S_{T,i}$ through the set of equations
\begin{equation}
    \begin{split}
        \nabla n_i &= - n_i k_{T,i} \nabla \ln T\\
        x_i \nabla n + n \nabla x_i &= - \frac{n_i k_{T,i}}{T} \nabla T\\
        x_i \lsp \ppder{n}{T}_{p,\vx}\nabla T + \sum_j \ppder{n}{x_j}_{T,p} \nabla x_j \rsp + n \nabla x_i &= - \frac{n_i k_{T,i}}{T} \nabla T\\
        - \frac{x_i T}{n_i} \ppder{n}{T}_{p,\vx} + \sum_j \lsp n \delta_{i,j} + \ppder{n}{x_j}_{T,p} \rsp x_j(1 - x_j) S_{T,j} &= k_{T,i}.
    \end{split}
\end{equation}

Alternatively, starting from Equation \eqref{eq:mflux_com_expanded} and setting $J_i = 0$, the Soret coefficients may be related directly to the Sonine polynomial expansion coefficients through the set of equations

\begin{equation}
    \begin{split}
        \sum_\ell S_{T,\ell} \sum_j \sum_k d_{i,j}^{(0)}E_{jk}\lsp n \delta_{\ell, k} + x_k \ppder{n}{x_{\ell}}_{T,p}\rsp & x_{\ell} (1 - x_{\ell}) \\
        = \sum_j \sum_k d_{i,j}^{(0)} E_{jk}x_k \ppder{n}{T}_{p,\vx} -& \frac{n}{T}\lsp a_i^{(0)} - \sum_j d_{i,j}^{(0)} \sum_k x_j\lrp \delta_{j,k} + \frac{4\pi}{3}n_kM_{jk} \sigma_{jk}^3 \chi_{jk} \rrp \rsp
    \end{split}
\end{equation}