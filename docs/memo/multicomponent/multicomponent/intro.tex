\section{Introduction}

The starting point for developing a kinetic model for multicomponent, high density mixtures is the same as that for the binary, single component case, with a minor modification. The Boltzmann equations for an $N$ component mixture may be written as

\begin{equation}
    \lsp \pder{}{t} + \vu_i \cdot \nabla + \pfrac{\Vec{F}_i}{m_i} \cdot \pder{}{\vu_i} \rsp f_i = \sum_j J_{ij}(f_i f_j), \hspace{.5cm} i = \{1, 2, ..., N\}
    \label{eq:boltzmann_multi}
\end{equation}

where $t$ is the time, $\vu_i$ is the velocity, $\Vec{F}_i$ is the sum of external forces, $m_i$ is the mass and $f_i$ is the velocity distribution function (vdf.) of species $i$. $J_{ij}$ is the streaming operator which becomes
\begin{equation}
    J_{ij}(f_i f_j) \equiv \int \int \int \chi_{ij}(\Vec{r}, \Vec{r} + \sigma_{ij}\hat{k}) f_i'(\Vec{r})f_j'(\Vec{r} + \sigma_{ij}\hat{k}) - \chi_{ij}(\Vec{r}, \Vec{r} - \sigma_{ij} \hat{k})f_i(\Vec{r})f_j(\Vec{r} - \sigma_{ij}\hat{k}) b\d b \d \epsilon \d \vu_j
    \label{eq:streaming_op}
\end{equation}

where $\hat{k}$ is the unit vector connecting the two particles, $b$ is the impact parameter and $\epsilon$ is the angular coordinate in the plane of $b$. The prime in $f_i'$ denotes functions of the post-collision velocities. In the same manner as for low-density mixtures, the streaming operator describes the rate of change in the vdf. of species $i$ due to collisions with species $j$. 

The modification when comparing to the low-density streaming operator is the introduction of the factor $\chi_{ij}$, the pair distribution function of the particles, which modifies the probability of finding particles $i$ and $j$ at positions $\Vec{r}_i$ and $\Vec{r}_j$. Furthermore, the vdf. of particle $j$ in the integral of Equation \eqref{eq:streaming_op} is evaluated at $\Vec{r} \pm \sigma_{ij} \hat{k}$ rather than at $\Vec{r}$. Here, $\sigma_{ij}$ is taken to be the distance between the centre of mass of the particles ''at contact''. For hard spheres, this definition is unproblematic but for particles interacting with some realistic potential the definition of being ''at contact'' is slightly less clear. For now, $\sigma_{ij}$ may be regarded as a parameter in the range of the molecular sizes, that is independent of particle velocity and position.

Following the Enskog solution method,\cite{cohen_1} de Haro et al. find that a first approximation to the vdf. may be written as
\begin{equation}
    f_i^{(1)} = f_i^{(0)}\lsp 1 + \Phi_i \rsp
\end{equation}

where 

\begin{equation}
    f_i^{(0)} = n_i \pfrac{m_i}{2 \pi k_B T}^{\frac{3}{2}} \exp \lsp - \sU_i^2 \rsp
    \label{eq:eq_vdf}
\end{equation}

is the Maxwell distribution function, with the peculiar velocity $\vU_i \equiv \vu_i - \vu^m$ defined relative to the \textit{centre of mass} velocity $\vu^m$ and the dimensionless peculiar velocity defined as $\sU_i^2 \equiv \frac{m_i}{2 k_B T}U_i$. $n_i$ is used to denote the particle density of species $i$.

Equivalently to the low-density case, $f_i^{(0)}$ satisfies the conservation equations of mass, energy and momentum exactly. That is,
\begin{equation}
    \begin{split}
        \int f_i^{(0)} \d \vu_i &= n_i, \hspace{.25cm} \forall \hspace{.25cm} i\\
        \sum_i \int f_i^{(0)} m_i \vu_i \d \vu_i &= \rho \vu^m \\
        \sum_i \int f_i^{(0)} \frac{m_i}{2} \vU_i ^2 \d \vu_i &= \frac{3}{2} n k_B T
    \end{split}
\end{equation}

where $\rho$ denotes the mass density of the mixture. Thus, for all $r > 0$ we can require that
\begin{equation}
    \begin{split}
        \int f_i^{(r)} \d \vu_i &= 0, \hspace{.25cm} \forall \hspace{.25cm} i\\
        \sum_i \int f_i^{(r)} m_i \vu_i \d \vu_i &= 0 \\
        \sum_i \int f_i^{(r)} \frac{m_i}{2} \vU_i ^2 \d \vu_i &= 0.
    \end{split}
\end{equation}

The equation of conservation of momentum is obtained by multiplying Equation \eqref{eq:boltzmann_multi} by $m_i \vu_i$. Reordering this equation and inserting for $f_i^{(0)}$, one can identify the hydrostatic pressure as

\begin{equation}
    p = p^k + p^\phi, \hspace{.5cm} p^k = n k_B T, \hspace{.5cm} p^\phi = \frac{2 \pi}{3} n^2 k_B T \sum_i \sum_j x_i x_j \sigma_{ij}^3 \chi_{ij}
\end{equation}

where $x_i$ denotes the mole fraction of species $i$.

In determining the first order approximation to the vdf. it is found that $\Phi_i$ is of the form
\begin{equation}
    \Phi_i = - \frac{1}{n} \vA_i \nabla \ln T - \frac{1}{n} \mB_i : \nabla \vu^m + \frac{1}{n} H_i \nabla \cdot \vu^m - \frac{1}{n} \sum_j \vD_i^{(j)} \vd_j'
\end{equation}

where $\vd_j'$ is defined by

\begin{equation}
    \vd_i = \sum_{j \neq i} \omega_j \vd_i' - \omega_i \vd_j'
\end{equation}

with $\omega_i$ denoting the weight fraction of species $i$ and
\begin{equation}
    \vd_i = - \frac{\rho_i}{\rho n k_B T} \lsp \nabla p + \sum_j \rho_j \lrp \frac{\Vec{F}_i}{m_i} - \frac{\Vec{F}_j}{m_j}\rrp \rsp + \sum_j x_i \lrp \delta_{i,j} + \frac{4 \pi}{3} n_j M_{ij} \sigma_{ij}^3 \chi_{ij}\rrp \nabla \ln T + \frac{x_i}{k_B T} \nabla_T \mu_j
    \label{eq:diffusion_driving_force}
\end{equation}

where $\delta_{i,j}$ is the Kronecker delta and $M_{ij} = \frac{m_i}{m_i + m_j}$. The final term, the gradient in chemical potential at constant temperature may be rewritten as

\begin{equation}
    \frac{x_i}{k_B T} \nabla_T \mu_i = \frac{x_i}{k_B T} \sum_j \ppder{\mu_i}{n_j}_{T,n_{k \neq j}} \nabla n_j
\end{equation}

yielding
\begin{equation}
    \vd_i = - \frac{\rho_i}{\rho n k_B T} \lsp \nabla p + \sum_j \rho_j \lrp \frac{\Vec{F}_i}{m_i} - \frac{\Vec{F}_j}{m_j}\rrp \rsp + \sum_j x_i \lrp \delta_{i,j} + \frac{4 \pi}{3} n_j M_{ij} \sigma_{ij}^3
    \chi_{ij}\rrp \nabla \ln T + \frac{1}{n} E_{ij} \nabla n_j
    \label{eq:diffusion_driving_force2}
\end{equation}

where one should recall that $n_i$ denotes the particle \textit{density} of component $i$.

The response functions $\vA_i$, $\mB_i$, $H_i$ and $\vD_i^{(j)}$ are related to the thermal conductivity, shear viscosity, bulk viscosity and diffusion coefficient of the mixture. In the same manner as for a dilute mixture, one may determine the transport coefficients by writing the response functions as polynomial expansions in the Sonine polynomials, and requiring that these expansions obey the constrains posed by the summational invariants.

In the following sections the resulting equations for the transport coefficients, and their relation to the fluxes will be given. In the case of diffusion, the matter of how the diffusion coefficient should be defined will be addressed.
